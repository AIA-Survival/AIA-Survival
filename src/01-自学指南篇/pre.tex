\part{自学指南篇}

\chapter{自学方法指南}

作为一个自学者(科研人,cs狗),往往在学习途中会遇到大量问题,因此你需要具备获取其它参考资料的能力。但在此之前, 你需要适应查阅英文资料。因为当你的学习愈发深入时, 你会发现你不太容易搜索到相关的中文资料。
那么何如适应查找英文资料?方法是 \textbf{尝试并坚持查阅英文资料}。

为了查找英文资料,你应该使用下表中推荐的网站:

\begin{table}[H]
	\centering
\begin{tabular}{|l|l|l|l|}
\hline
      & 搜索引擎           & 百科                      & 问答网站                                                                                  \\ \hline
推荐使用  & www.google.com & http://en.wikipedia.org & http://stackoverflow.com                                                              \\ \hline
不推荐使用 & www.baidu.com  & http://baike.baidu.com  & \begin{tabular}[c]{@{}l@{}}http://zhidao.baidu.com\\ http://bbs.csdn.net\end{tabular} \\ \hline
\end{tabular}
\end{table}

\begin{refbox}
一个素质合格的CSer需要具备的另一个标准是, 懂得如何提问.

相信大家作为CSer, 被问如何修电脑的事情应该不会少. 比如你有一个文科小伙伴, 他QQ跟你说一句"我的电脑出问题了", 让你帮他修. 然后你得问东问西才了解具体的问题, 接着你让他尝试各种方案, 让他给你尝试的反馈. 如果你有10个这样的小伙伴, 相信你肯定受不了了.

事实上, 如果希望能提高得到回答的概率, 提问者应该学会如何更好地提问. 换句话说, 提问者应该去积极思考 "我可以主动做些什么来让对方更方便地帮助我诊断问题". 文科小伙伴确实不是学习计算机专业的, 你可以选择原谅他; 但你是CSer, 至少你得在问题中描述具体的现象以及你做过的尝试, 而不是直接丢一句"我的程序挂了", 就等着别人来救场. 在你将来的职业生涯中也很有可能需要向别人求助, 比如在github等开源社区中发issue, 或者是在stackoverflow等论坛上发帖, 或者给技术工程师发邮件等, 如果你的提问方式非常不专业, 很可能没有人愿意关注你的问题, 因为这不仅让人觉得你随便提的问题没那么重要, 而且大家也不愿意花费大量的时间向你来回地咨询.

一种推荐的提问方式如下:

我在xxx的时候遇到了xxx的错误. 这个错误可以通过以下步骤重现: (描述具体的现象)\\
1. 我的系统版本是xxx, 相关的工具版本是xxx\\
2. 我做了xxx (必要的时候贴个图)\\
3. 然后xxx (必要的时候贴个图)\\
...\\
为了排查这个错误, 我进行了以下尝试: (说明我很希望可以解决问题, 真的没办法才提问的)\\
1. 我做了xxx, 出现了xxx的结果 (必要的时候贴个图)\\
2. 我还做了xxx, 出现了xxx的结果 (必要的时候贴个图)\\
...\\
最后问题还没有解决, 请问我还需要做哪些事情?\\

另外请大家务必阅读\href{https://github.com/ryanhanwu/How-To-Ask-Questions-The-Smart-Way/blob/main/README-zh_CN.md}{提问的智慧}和\href{https://github.com/tangx/Stop-Ask-Questions-The-Stupid-Ways/blob/master/README.md}{别像弱智一样提问}这两篇文章, 里面有不少例子供大家参考.
\end{refbox}

\textbf{STFW \& RTFM}

\begin{refbox}
你或许会想, 我问别人是为了节省我的时间.

但现在是互联网时代了, 在网上你能得到各种信息: 比如diff格式这种标准信息, 网上是100\%能搜到的; 就包括你遇到的问题, 很大概率也是别人在网上求助过的. 如果对于一个你本来只需要在搜索引擎上输入几个关键字就能找到解决方案的问题, 你都没有付出如此微小的努力, 而是首先想着找人来帮你解决, 占用别人宝贵的时间, 你将是这个时代的失败者.

于是有了STFW (Search The F**king Web) 的说法, 它的意思是, 在向别人求助之前自己先尝试通过正确的方式使用搜索引擎独立寻找解决方案.

正确的STFW方式能够增加找到解决方案的概率, 包括

\begin{itemize}
	\item 使用Google搜索引擎搜索一般性问题
    \item 使用英文维基百科查阅概念
    \item 使用stack overflow问答网站搜索程序设计相关问题
\end{itemize}

如果你没有使用上述方式来STFW, 请不要抱怨找不到解决方案而开始向别人求助, 你应该想, "噢我刚才用的是百度, 接下来我应该试试Google". 关于使用Google, 可以尝试设置"科学上网", 具体设置方式请STFW.

\vskip 1ex
\hrule
\vskip 1ex

RTFM是STFW的长辈, 在互联网还不是很流行的年代, RTFM是解决问题的一种有效方法. 这是因为手册包含了查找对象的所有信息, 关于查找对象的一切问题都可以在手册中找到答案.

你或许会觉得翻阅手册太麻烦了, 所以可能会在百度上随便搜一篇博客来尝试寻找解决方案. 但是, 你需要明确以下几点:

\begin{itemize}
    \item 你搜到的博客可能也是转载别人的, 有可能有坑
    \item 博主只是分享了他的经历, 有些说法也不一定准确
    \item 搜到了相关内容, 也不一定会有全面的描述
\end{itemize}

最重要的是, 当你尝试了上述方法而又无法解决问题的时候, 你需要明确"我刚才只是在尝试走捷径, 看来我需要试试RTFM了".

\end{refbox}

\begin{titledbox}{参考资料}
	\href{https://nju-projectn.github.io/ics-pa-gitbook/ics2021/index.html#%E6%90%9C%E7%B4%A2%E5%BC%95%E6%93%8E-%E7%99%BE%E7%A7%91%E5%92%8C%E9%97%AE%E7%AD%94%E7%BD%91%E7%AB%99}{搜索引擎, 百科和问答网站}
\end{titledbox}
