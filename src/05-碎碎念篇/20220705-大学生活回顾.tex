\subsubsection{写于大三最后一门课考完后的大学生活回顾}

\textbf{碎碎念的原因}

今天是 7 月 5
号,刚刚考完计算机控制技术,我有种大学生活在此结束的感觉,因为以后不用再为加权而卷,能够有更多的时间去做自己真正想做的事情。

\textbf{01 大一上}

上大学前,我就对科研感兴趣,觉得自己在大学一定要有一段像样的科研经历(很遗憾,现在没有)

进入大学后,我百分百延续了高中时的思维,即加权至上。当时的我认为,大学的学习模式和高中没有区别,``最优解''都是好好听讲,好好写作业,最后考出一个理想的分数。而所谓科研,正如高中时候的社团,是课内学习的``佐料'',不应当占用大量的时间。大一上时的我没有进行任何课外的学习探索,只是在被动地去接受学院给我们提前安排好的路线------《培养计划》。

\textbf{02 大一下}

遭遇了疫情,在家上网课。这一学期我报了魏巍老师的《信息检索》公选课,这是我接触
NLP
相关知识的开始,也是投入较多时间去学习课外知识的开始。在这学期末的时候,我向魏巍老师申请加入他的实验室,初步踏上了科研的道路。与大一上相比,大一下的我往课外科研中开始投入了时间(但不多,还是课内为大头)。这段不长的``科研经历''让我开始思考,跟随《培养计划》的路线是否合理。

且不说我们的《培养计划》对于以\emph{传统自动化行业}为目标的同学是否合适,每个人都是不同的,都有与其他人不一样的兴趣、理想与追求,《培养计划》不可能适用于所有的同学。我不能局限在学院给我们设置好的《培养计划》中。《培养计划》固然是死的,但我可以自己改变,比如:往课内少投入一些时间,往课外的科研中多投入一点时间。

这也有一个矛盾:加权和科研并不是相辅相成的,如果将更多精力投入到科研中,课内的加权必然会有所下降;如果想要稳住加权,那就不能往科研中投入太多时间。我也对此进行了一段时间的考虑,最后觉得不能让加权这样一个冰冰凉凉的数字成为我大学
4 年的主基调。相比于做了自己想做的事情(于我而言是科研)但加权不高,浪费
4 年时间只取得一个不错的分数更不能让自己接受。我不想在大学毕业之后回望 4
年生活时,发现自己除了一个优秀的分数一无所有。我不想以后对我的大学生活做一个自我评价时,用的是加权这样一个数字作为指标,而不是除去加权之外剩余的东西。

\textbf{03 大二上}

我把重头从课内挪到了科研,频繁逃课去实验室学习。在这一学期里,我在阅读论文、做实验中两头徘徊。这一学期比较遗憾的就是想独立完成一篇论文,而没有让学长、学姐带着做,因此走了不少弯路。比如第一次和魏老师讨论的时候是空着手去的(指没准备
PPT
之类的任何东西),以为科研中的讨论是和上课一样的,都是老师讲,我回答;读完的论文不整理,不做笔记,一篇篇论文的
PDF
文档散落在电脑的各个角落中。不过在自己的探索中,我也逐渐发现了自己的问题,并开始调整自己的习惯与思维方式,比如会将所有阅读过的论文重命名后放在一个文件夹下(现在感觉
zotero
才是yyds)。在这个学期中,我在科研上投入了大量的时间,没有什么产出,但在独立探索的过程中仍然收获颇丰。

\textbf{04 大二寒假}

有一次去书城看书,遇到了一本叫做《CTF
竞赛权威指南-PWN篇》的书。我有点好奇 CTF
竞赛是个什么东西,所以把这本书拿了起来。在翻阅过程中,我发现它非常的有趣(讲了比较
foundamental
的知识),于是把这本书买了回家,也萌生了一丝想去做安全的念头。在寒假中,我还是在做
NLP 的事情,但我也在不断思考着是否要转去``安全''。

这个选择对我是有些困难的:一边是有一些基础,刚刚踏上科研道路的
NLP;一边是虽然并不特别了解,但我觉得更加感兴趣的安全。如果转去做安全,估计到大四也没法有一段完整的科研经历,这在保研的时候毫无疑问是不利的。我最后还是决定选择自己的兴趣,转到安全领域去,因为我觉得越早转变方向越容易,越往后的话,淹没成本越高。我决定完成手上的
NLP 工作后就完全转向安全。

【种一棵树最好的时间是十年前,其次是现在】

\textbf{05 大二下}

我冲着网安组加入了 Dian 团队,然后频繁逃课去 Dian 团队学习。虽然我在
Dian 团队呆的时间不长,在团队里的时候也经常摸鱼,但 Dian
团队的确教会了我许多东西。以前的我几乎不看官方文档,遇到问题只会百度和
CSDN;Dian 团队教会了我文档的重要性,教会了我扔下 CSDN 转向 Stack
Overflow。Dian
团队的经历极大地提升了我解决实际问题的能力。但由于对课内学习过分的不在意,这学期我加权炸了。86
分的加权提醒我大三的时候要转变一下学习模式,不然保研资格丢了就有些寄了(现在感觉考虑出国是更优解)。

大二暑假,做 NLP 未完成的工作。在 9 月份投稿 AAAI。

\textbf{06 大三}

大三上时,我收到了 AAAI
的拒稿。但一方面我要稳一下加权,一方面我需要投入时间到安全中去,对被拒稿的论文做修改的想法就被搁置下来了。这学期我考虑的比较多的是研究生的事情:我想到外校读研,但我没有任何科研的成果(除了一篇拒稿的文章),也没有安全方面的积累,想要通过夏令营/预推免的选拔是困难的。于是我主动联系了张源老师,并开始做
Fuzzing 相关的工作。

大三下我基本延续了大三上的做法:一边稳定加权,一边做 Fuzzing
的工作。这学期还算顺利。

\textbf{07 垃圾话}

在整段大学生活中,我没有什么产出,但我确实在不断的探索与思考。在大二下/大三上时我才掌握了学习新的领域、解决未知问题的方法;在大三下的时候才找到一个适合自己的课内与课外的平衡。在大学三年,我做过不少错误的选择(比如过度执着于保研),但都无妨,从这些错误中我也学会了许多。

希望夏令营和保研一切顺利。

\flushright{by lza}
